\section{Validation}
We use the following constants:
\begin{align}
d_\text{max}^\text{transition} &= \SI{1}{\milli\meter} \\
d^\text{discretization} &= \SI{0.2}{\milli\meter} \\
t_\text{beading} &= \SI{0.4}{\milli\meter} \\
d_\text{max}^\text{intersection} &= \SI{75}{\percent}
\end{align}

We use a transition anchor position of
%$$t_-(n) =  t(n) \frac{ b^{-1}(n) - p(n) }{p(n+1) - p(n) }$$
%$$t_-(n) =  t(n) \frac{ b^{-1}(n) - 0.4n }{0.4(n+1) - 0.4n }$$
%$$t_-(n) =  t(n) \frac{ b^{-1}(n) - 0.4n }{0.4}$$
$$t_0(n) =  t(n) \left( b^{-1}(n) / 0.4  - n \right)$$
This should ensure that the transitions never overlap with the locations $v$ where $2 R(v) = 0.4n$ for $n \in \mathbb{N}$, since \SI{0.4}{\milli\meter} is the prefered bead width for a similar nozzle size.

Our beading strategies are based on a prefered width $w_\text{pref} = \SI{0.4}{\milli\meter}$, which is equal to the size of the hole in the printing nozzle.

We set 
\begin{align*}
t(n) &= w_\text{pref} \\
\alpha_\text{max} &= \SI{135}{\degree} \\
L(n,d)_i &= 
\begin{cases}
-\frac12 W(n,d)_i + \sum_{j=0}^i W(n,d)_j & \text{ if } i < \frac12 (n -1) \\
d/2 & \text{ if } i =  \frac12 (n -1) \\
d - W(n,d)_{n-1-i} & \text{ otherwise }\\
\end{cases}
\end{align*}

\subsection{Strategies}
We can emulate a variety of toolpath generation methods from related literature by applying various beading strategies to our framework.

\paragraph{Naive Strategy}
Because the naive strategy never employs single line polyline segments, we disable marking edges using the significance measure by choosing the limit angle.
\begin{align*}
\alpha_\text{max} &= \SI{180}{\degree} \\
b(d) &= 2 \left\lfloor \frac{d}{ 2w_\text{pref}} + \frac12 \right\rfloor \\
W(n,d)_i &= w_\text{pref} \text{ for all } i 
%\\
%L(n,d)_i &= w_\text{pref} \left(i + \frac12 \right) \text{ for all } i < \frac12 n
\end{align*}



\paragraph{Constant bead count}
Emulates \cite{Ding2016a}

\begin{align*}
b(d) &= C \\
W(n,d)_i &= d / n \text{ for all } i 
%\\
%L(n,d)_i &= d / n \left(i + \frac12 \right) \text{ for all } i < \frac12 n
\end{align*}

\paragraph{Deviating toolpath at center}
Emulates \cite{Jin2017}

$r_\text{min} = 0.8 w_\text{pref}$ and $r_\text{max} = 1.25 w_\text{pref}$

\begin{align*}
b^-(d) &= 2 \left\lfloor \frac{d}{ 2w_\text{pref}} + \frac12 \right\rfloor \\
b(d) &= b^-(d) +
\begin{cases}
-1 & \text{ if } b^-(d) w_\text{pref} - d > w_\text{pref} - r_\text{max} \\
1  & \text{ if }  b^-(d) w_\text{pref} - d < w_\text{pref} - r_\text{min} \\
0 & \text{ otherwise}
\end{cases}
\\
W(n,d)_i &= 
\begin{cases}
d - (n-1) w_\text{pref} &\text{ if } i = \frac12 (n-1) \\
w_\text{pref} &\text{ otherwise }
\end{cases}
%\\
%L(n,d)_i &= 
%\begin{cases}
%d / 2 & \text{ if } i = \frac12 (n-1) \\
%w_\text{pref} \left(i + \frac12 \right) & \text{ otherwise }
%\end{cases}
\end{align*}



\paragraph{Only outer bead}
Emulates \cite{Moesen2011}

$d_\text{max}^\text{intersection} = \SI{0}{\percent}$

\begin{align*}
t(n) &= 0 \\
b(d) &=
\begin{cases}
1 & \text{ if } d < w_\text{pref} \\
2 & \text{ otherwise } \\
\end{cases}
 \\
W(n,d)_i &= 
\begin{cases}
d & \text{ if } n = 1 \\
w_\text{pref} & \text{ otherwise } \\
\end{cases}
%\\
%L(n,d)_i &= 
%\begin{cases}
%d / 2 & \text{ if } n = 1 \\
%w_\text{pref} / 2 & \text{ otherwise } \\
%\end{cases}
\end{align*}


\paragraph{Distributed strategy}
Distribute the overfill or underfill which would happen using a naive strategy over all beads.
This maximizes robustness and minimizes narrow beads qhich are difficult to print.


\begin{align*}
b(d) &= \left\lfloor \frac{d}{ w_\text{pref}} + \frac12 \right\rfloor \\
W(n,d)_i &= d / n \text{ for all } i 
%\\
%L(n,d)_i &= d / n (i + \frac12) \text{ for all } i < \frac12 n
\end{align*}


\paragraph{General distributed strategy}
Distribute the overfill or underfill which would happen using a naive strategy $E$ over all beads unevenly depending on some weighing function $M(n.d)$, which defines the portion of overfill or underfill which would occur in the naive strategy to distribute to each junction.


\begin{align*}
b(d) &= \left\lfloor \frac{d}{ w_\text{pref}} + \frac12 \right\rfloor \\
E(n,d) &= d - n w_\text{pref} \\
W(n,d)_i &= w_\text{pref} + E(n,d) \frac{M(n,d)_i}{\sum_{j=0}^{n-1} M(n,d)_j} \text{ for all } i 
%\\
%L(n,d)_i &= d / n (i + \frac12) \text{ for all } i < \frac12 n
\end{align*}


\todo{TODO: check $n$ vs $n-1$ everywhere when comparing indices and the total size! Largest index should be $n-1$, not $n$!}

\paragraph{Inward distributed strategy}
For example, we can choose 
$$M(n,d)_i = \max(0, 1 - \frac{1}{N^2} (i - (n-1)/2)^2 )$$
To distribute the hypothetical over-/underfill over the innermost $2N$ beads, and distribute most of it to the inner beads.
That way we limit the impact of the distributed strategy and have the nominal bead width $w_\text{pref}$ in regions farther away from the middle.
This helps if print settings for bead widths other than the prefered width aren't calibrated as well as the settings for $w_\text{pref}$.
Toolpaths with deviating widths might result in less dimensional accuracy on the outline and uncalibrated mechanical properties throughout the volume, so limiting the number of beads width deviating bead width might be beneficial.
\todo{Emphasize this point more. Put it in the abstract and conclusion.}






\subsection{Computational analysis}
\paragraph{Accuracy}
Render toolpaths and calculate amount of overfilling and amount of underfilling.

\paragraph{Printability}
Define some function to determine printability $0<P<1$, 
then define a loss function $L = 1/P$.
Evaluate average loss divided by total toolpath length.

\old{Visualize thickness of beads as color.}

Show results for different settings.

\old{Also render with nozzle size as minimal width and use middle of toolpaths instead of middle of beads.}

Example shapes:
\begin{itemize}
\item same as Moessen: grids of triangular holes
\item same as Moessen: circular holes.
\item same as Jin 
\item same as Kao
\item wedge
\item layer of a typical model: phone case
\item some of the above shapes, but with some fuzzy randomized outline
\end{itemize}





\subsection{Experimentation}
Print objects and show qualities:
\begin{itemize}
\item visual consistency of flat top skin surface
\item visual gradual transparency changes
\item graphs of tensile tests on thin walled object
\end{itemize}

Print on mississipi?

What material should I print with for the tensile tests?









