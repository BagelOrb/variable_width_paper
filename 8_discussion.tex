%\section{Discussion}


\subsection{Comparison of beading schemes}
We can see from \cref{TEST_naive_accuracy}(top) and~\ref{over_underfill} that the uniform technique causes a lot of overfills and underfills: on average  {\SI{1.6}{\percent}} of the total target area is covered by underfill and likewise for overfill.
To our knowledge, the uniform beading scheme, as well as the outer beading scheme, is of little use to FDM printers.

The constant bead count scheme effectively deals with underfills, but generates orders of magnitude more overfills compared to the other schemes. 
Also, the scheme comes at the cost of greatly varying bead widths and an average bead width that is not close to the preferred bead width.
Note that most overfill areas occur near regions of alternating bead width. 
{While the scheme results in short toolpaths, as indicated by the idealized print time, it also results in a wide range of bead widths, which cause the back pressure compensation print time to be very large.
See \cref{statisticsfig}.}
For an input outline shape which contains both very small and very large features, the constant bead count scheme produces bead widths which can fall outside of the range of manufacturable bead widths.
Moreover the centrality marking is not robust against small perturbations in the outline; adding a small chamfer in a corner causes the unmarked ST to be very small at that location, which results in tiny bead widths.
{See top right of \cref{TEST_Constant_accuracy}.}

In \cref{TEST_Center_accuracy} we can see that
the centered beading scheme effectively deals with overfill and produces desired bead widths in all locations, except for the extrusion paths in the center, where the bead widths {range between $0.5 w^*$ and $1.8w^*$, i.e. the variation is within a factor of $3.6$.}
{However, it does produce some narrow underfill regions.}
{
Compared to the uniform technique the centered technique increases the (open) path count, but considerably reduces over- and underfill and decimates the number of toolpath angles below \SI{45}{\degree}.
See \cref{statisticsfig}.
}

However, according to \cref{widthHistogram} the centered scheme exhibits a wider range of bead widths than the distributed schemes:
the standard deviation of the bead widths in the centered scheme is approximately \SI{39}{\micro\meter}, while that of the distributed schemes is approximately \SI{14}{\micro\meter}.%_
{\footnote{{Although the standard deviation $\sigma$ of the inward distributed scheme is slightly higher, the mean absolute deviation is lower than that of the evenly distributed scheme (i.e. \SI{9}{\micro\meter} versus \SI{11}{\micro\meter}), because its distribution is more peaked.}}}
{Moreover, because the quantization operator rounds to the nearest number of beads, in the worst case where we switch from a single to two beads the widths switch from $0.75w^*$ to $1.5w^*$, i.e. the variation is within a factor $2$, which is considerably lower the than factor of $3.6$ in the centered scheme.}
We therefore conclude that the distributed schemes {exhibit a lower bead width variation and lower (open) path count} compared to the centered scheme.
%This is desirable for the manufacturability of the beads and can therefore have a positive effect on the mechanical properties and surface quality of the 3D prints. 


{\Cref{distributed_comparison,TEST_Distributed_accuracy,TEST_InwardDistributed_accuracy} show that in the inward distributed scheme the outer toolpaths have the preferred width more often than in the evenly distributed scheme, which means that the outline accuracy of the inward distributed beading is less affected by inaccuracies in the adaptive width control system.}
%{\Cref{TEST_InwardDistributed_accuracy} shows that in the inward distributed scheme the outer beads are more often equal to the preferred width compared to the evenly distributed scheme in \cref{TEST_Distributed_accuracy}, but that effect is more pronounced for wider geometry such as in \cref{distributed_comparison}.}
%While the difference between the evenly and inward distributed scheme in \cref{visualized_accuracy} results only in the outer bead being the preferred width in some locations, the effect of the inward distributed scheme is more pronounced in larger geometry, as can be seen in \cref{distributed_comparison}.}
 % _
Furthermore, we find that compared to evenly distributed, the inward distributed scheme produces {less corners with angles above \SI{130}{\degree} and less overfill, because the area of influence that bead count transitions have is limited in the inward distributed scheme}.
Thus the inward distributed scheme prevents over- and underfill, generates smooth toolpaths with more homogeneous width and affects smaller more centered parts of the print than the other schemes{, while incurring little to no extra print time}. 




{
\subsection{Limitations}
Because the performance of the various toolpathing techniques depends on the geometry of a model, they have ramifications for the practice of design for additive manufacturing.
Because the naive method produces under- or overfill for parts of an outline with a constant diameter $d \neq 2 i w^*$ it is best practice to design a model such that horizontal cross-sections have a feature diameter of an even integer multiple $i$ of the bead width.
For the center scheme and for the current state of the art one should only avoid parts for which $(2i + 1.8) w^* < d < (2i + 0.5) w^*$ in order to avoid underfill.
For the distributed schemes however, there is no diameter at which the framework produces under- or overfill for a part with a constant diameter $d$.
The design consideration therefore reduces to limiting the diameter of your parts to be within the range $[w_\text{min}, \infty)$,
where $w_\text{min}$ is a configurable parameter when using the widening meta-scheme.


The default limit bisector angle $\alpha_\text{max} = \SI{135}{\degree}$ ensures that we don't employ transitioning in shallow wedge regions, which would result in a lot of short odd single bead polylines, which would break up the semi-continuous nature of polygonal extrusion paths;
$\alpha_\text{max} = \SI{135}{\degree}$ corresponds to $w^* / \cos \nicefrac12 \alpha_\text{max} \approx \SI{0.4}{\milli\meter}$ long segments
and under-/overfill areas of $\nicefrac14 (w^*)^2 \left( \tan ( \alpha / 2) - \alpha / 2 \right) \approx \SI{0.05}{\milli\meter\squared}$.
However, future work might be aimed at reducing under-/overfill in regions with a low bisector angle without the introduction of short single polyline extrusion segments.
If the over-/underfill problem is also solved for non-significant regions we might be able to increase $\alpha_\text{max}$ and reduce the discontinuity introduced by short extrusion segments.

Another limitation of our method is that in a location $v$ with locally maximal $R(v) \approx (i + \nicefrac12) w^*$ the odd bead count will result in a single polyline extrusion segment consisting of only a single point.
This can be viewed in the bottom right of \cref{TEST_InwardDistributed_accuracy} for example.
In order to print such a dot, we make it into a \SI{10}{\micro\meter} long extrusion segment, with an altered width such that the total volume remains correct.
A more principled way of dealing with such a situation remains future work.

Finally it should be noted that although our framework can accurately emulate the constant bead count approach by \citeauthor{Ding2016a}, its emulation of the centered approach by \citeauthor{Jin2017JMS} is imperfect.
The transitions resulting from out framework introduce sharper corners and there is more width variation in those corners.
Whereas the width of the connecting segment in the approach by \citeauthor{Jin2017JMS} is the preferred width $w^*$, the bead widths closer to the center resulting from our framework will be twice the local radius, which is larger than $w^*$.
However, this inflated bead width variation is expected to have an insignificant impact on the measured bead width variation.
}



\subsection{Applications}
Toolpath\revise{}{s} with varying width is particularly meaningful for narrow parts, since there the negative effect of under- and overfill is more pronounced than in wide parts.
In extreme cases, thin features will not be filled at all.
Therefore, our framework, while working for wide parts as well, shows most of its potential for objects which contain thin parts.

\Cref{applications_overview} collectively shows the application of the proposed inward distributed scheme for various types of 3D model, including both thin parts (architectural models, casings, embossed text, gears and microstructures) and wide parts (\cref{applications_case}) and organic shapes (\cref{applications_statue})).

For architectural models and casings, preventing over- and underfill is expected to make them stronger. 
For embossed text, preventing underfill reduces the various holes in the top surfaces, which is detrimental to the visual quality of those top surfaces.
For gears and similar mechanical parts that are designed with finite element analysis, the less variation in extrusion widths is closer to the assumptions under fast analysis (e.g. using homogenization~\cite{Liu2016CAD}).

Of particular interest are microstructures that could be uniquely fabricated by 3D printing.
For example, topology optimized bone-like structures~\cite{wu2017infill} contain filaments of varying thickness that follow a varying stress direction (\cref{applications_bone}).
An angled Gyroid structure with uniform thickness also results in outline shapes with varying width (\cref{applications_gyroid}). 
These structures are accurately densely filled using our framework.
Another class of microstructures consists of parameterized patterns with varying thickness to achieve functional gradation.
\Cref{applications_hex} shows the contour-parallel toolpath\revise{}{s} with varying width of a hexagonal grid neatly switches between different bead counts over the volume, preventing the jagged moves a direction-parallel toolpath\revise{}{s} would create for such a case~\cite{bates2018compressive}.


\begin{figure*}
\centering
\setlength{\figwidth}{0.099\textwidth}
\setlength{\figheight}{0.099\textwidth}
\begin{subfigure}{\textwidth}\centering
\includegraphics[width=\textwidth]{sources-applications-combined-small-dilated-circled.pdf}
%\caption{Overview}\label{applications_overview}
\end{subfigure}
\begin{subfigure}[t]{\figwidth}\centering
\includegraphics[height=\figheight]{sources-applications-house}
\caption{House}\label{applications_house}
\end{subfigure}
\begin{subfigure}[t]{\figwidth}\centering
\includegraphics[height=\figheight]{sources-applications-pocket-operator-case}
\caption{Case}\label{applications_case}
\end{subfigure}
\begin{subfigure}[t]{\figwidth}\centering
\includegraphics[height=\figheight]{sources-applications-david}
\caption{Statue}\label{applications_statue}
\end{subfigure}
\begin{subfigure}[t]{\figwidth}\centering
\includegraphics[height=\figheight]{sources-applications-tud-logo}
\caption{TUD}\label{applications_tud}
\end{subfigure}
\begin{subfigure}[t]{\figwidth}\centering
\includegraphics[height=\figheight]{sources-applications-ultimaker-logo}
\caption{UM}\label{applications_um}
\end{subfigure}
\begin{subfigure}[t]{\figwidth}\centering
\includegraphics[height=\figheight]{sources-applications-pinion-gear-motor}
\caption{Gear}\label{applications_gear}
\end{subfigure}
\begin{subfigure}[t]{\figwidth}\centering
\includegraphics[height=\figheight]{sources-applications-topopt-bone}
\caption{Bone}\label{applications_bone}
\end{subfigure}
\begin{subfigure}[t]{\figwidth}\centering
\includegraphics[height=\figheight]{sources-applications-gyroid}
\caption{Gyroid}\label{applications_gyroid}
\end{subfigure}
\begin{subfigure}[t]{\figwidth}\centering
\includegraphics[height=\figheight]{sources-applications-hex-grid}
\caption{Hex}\label{applications_hex}
\end{subfigure}
\begin{subfigure}[t]{.3\figwidth}
\includegraphics[height=\figheight]{sources-validation-widths-legend-small.pdf}
\end{subfigure}
\caption{
Visualization of the widths for the output toolpaths of the inward distributed beading scheme \revise{}{($N=3$) }applied to various example application objects.
From left to right and top to bottom: a house, a case for electronics, a statue, two common logos, a gear, a topologically optimized bone structure, a tilted homogeneous gyroid structure and a heterogeneous thickness hexagonal grid.
}
\label{applications_overview}
\end{figure*}













% Taking a broader perspective, we note that our proposed inward distributed scheme is a pragmatic solution.
% Rather than deriving some optimal beading scheme from a clear specification of the objective, we propose some arbitrary inward distributed beading scheme and show that it is better than the other beading schemes.
% An optimal beading scheme can be derived if the objective is formalized terms of a unambiguous fitness function, but that would depend on the specific hardware setup and application for which toolpaths are generated.
% This manuscript is therefore limited to showing the flexibility and versatility of the framework, rather than deriving an optimal beading scheme.




