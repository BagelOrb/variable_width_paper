\section{Discussion}

\subsection{Theoretical analysis}
In this paper the significance measure is not used heuristically, as it is sometimes used in other literature.
The significance measure is an exact measure of the relative size of gaps when using the naive toolpathing method.
\todo{[figure showing wedges, naive toolpaths and gaps]}
Note also that contrary to some literature we don't remove bones from the skeleton, but leave it intact.
The final skeleton is still an exact and full feature descriptor.
In that sense our framework provides an exact solution space.

Note also that our method is robust against small perturbations in the input shape.
Although the skeleton might acquire extra bones, the significant portions of the skeleton remain similar and the output toolpaths change infinitesimally.

Our framework is also local.
The toolpathing around some section of the input polygons is independent of far away regions of the polygon.
This means the toolpathing is stable against large perturbations of the input shape in far away regions
and it also allows for some parallelism during the computation of the toolpaths.


\subsection{Comparison of different beading strategies}
\todo{Discussion of finding of Validation section}