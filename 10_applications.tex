\section{Applications}

\subsection{Thin parts}
\begin{itemize}
\item Casings (e.g. for electronics)
\item Prototypes for blowmold parts or sheet metal parts etc.
\item Miniature models, architectural models
\item Microstructures
\item Lithophanes
\end{itemize}

Show the effect of the planned toolpath on the performance of varying thickness microstructures.
Compare to \cite{bates2018compressive}.


\begin{figure*}
\centering
\setlength{\figwidth}{0.1\textwidth}
\setlength{\figheight}{0.1\textwidth}
\begin{subfigure}{\figwidth}\centering
\includegraphics[height=\figheight]{sources/applications/david.png}
\caption{Statue}%label{applications_}
\end{subfigure}
\begin{subfigure}{\figwidth}\centering
\includegraphics[height=\figheight]{sources/applications/gyroid.png}
\caption{Gyroid}%label{applications_}
\end{subfigure}
\begin{subfigure}{\figwidth}\centering
\includegraphics[height=\figheight]{sources/applications/hex_grid.png}
\caption{Hex}%label{applications_}
\end{subfigure}
\begin{subfigure}{\figwidth}\centering
\includegraphics[height=\figheight]{sources/applications/house.png}
\caption{House}%label{applications_}
\end{subfigure}
\begin{subfigure}{\figwidth}\centering
\includegraphics[height=\figheight]{sources/applications/pinion_gear_motor.png}
\caption{Gear}\label{applications_gear}
\end{subfigure}
\begin{subfigure}{\figwidth}\centering
\includegraphics[height=\figheight]{sources/applications/pocket_operator_case.png}
\caption{Case}%label{applications_}
\end{subfigure}
\begin{subfigure}{\figwidth}\centering
\includegraphics[height=\figheight]{sources/applications/topopt_bone.png}
\caption{Bone}%label{applications_}
\end{subfigure}
\begin{subfigure}{\figwidth}\centering
\includegraphics[height=\figheight]{sources/applications/tud_logo.png}
\caption{TUD}%label{applications_}
\end{subfigure}
\begin{subfigure}{\figwidth}\centering
\includegraphics[height=\figheight]{sources/applications/ultimaker_logo.png}
\caption{UM}%label{applications_}
\end{subfigure}
\caption{
Closeups of results for the inward distributed beading strategy for various applications.
The extrusion widths are visualized exaggeratedly and using a color scheme exaplained in \cref{visualized_accuracy}.
}
\label{application_closeups}
\end{figure*}


\begin{figure}
\centering
\includegraphics[width=\columnwidth]{sources/applications/combined_small_dilated_circled.pdf}
\caption{
Visualization of the widths for the output toolpaths of the inward distributed beading strategy applied to various example application objects.
A legend for the colors can be found in \cref{visualized_accuracy}.
From left to right and top to bottom: a house, a case for electronics, a statue, two common logos, a gear, a topologically optimized bone structure, a homogeneous lateral thickness tilted gyroid structure and a heterogeneous thickness hexagonal grid.
}
\label{applications_overview}
\end{figure}


\begin{figure}
\centering
\includegraphics[width=\columnwidth]{sources/applications/wedge_print_v2.png}
\caption{
A small wedge shape printed using the variable width scheme prescribed by the ditributed beading strategy.
Small gaps have been added in between the extrusion paths so that the toolpaths can easily be identified separately.
}
\label{wedge_print}
\end{figure}


\subsection{Widening}
Regions where the model is narrower than the nozzle size can be printed with a bead width larger than the model thickness.