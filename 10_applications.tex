\subsection{Applications}
Toolpath with varying width is particularly meaningful for narrow parts, since there the negative effect of under- and overfill is more pronounced than in wide parts.
In extreme cases, thin features might not be filled at all.
Therefore, our framework, while working for wide parts as well, shows most of its potential for objects which contain thin parts.

\Cref{applications_overview} collectively shows the application of the proposed inward distributed scheme for various types of 3D model, including both thin parts (architectural models, casings, embossed text, gears and microstructures) and wide parts (larger regions (\cref{applications_case}) and organic shapes (\cref{applications_statue})).

For architectural models and casings, the prevention of over- and underfill is expected to make them stronger. 
For embossed text, the prevention of underfill reduces the various holes in the top surfaces, which is detrimental to the visual quality of those top surfaces.
Gears benefit from the prevention of under- and overfill regions and the distributed bead width discrepancy in that the mechanical properties of the toolpath are more consistent throughout. \jun{I don't follow the benefit for gears}

Of particular interest are microstructures that could be uniquely fabricated by 3D printing.
Microstructures which have layers with varying widths benefit from the prevention of under- and overfill, which will result in structures with mechanical properties closer to expectation.
For example, topology optimized bone-like structures~\cite{wu2017infill} contain filaments of varying thickness that follow a varying stress direction (\cref{applications_bone}).
An angled Gyroid structure with uniform thickness results in outline shapes with varying width (\cref{applications_gyroid}).
Another class of microstructures consists of parameterized patterns with varying thickness to achieve functional gradation.
\Cref{applications_hex} shows the contour-parallel toolpath with varying width of a hexagonal grid neatly switches between different bead counts over the volume, preventing the jagged moves a direction-parallel toolpath would create for such a case~\cite{bates2018compressive}.


\begin{figure*}
\centering
\setlength{\figwidth}{0.099\textwidth}
\setlength{\figheight}{0.099\textwidth}
\begin{subfigure}{\textwidth}\centering
\includegraphics[width=\textwidth]{sources/applications/combined_small_dilated_circled.pdf}
%\caption{Overview}\label{applications_overview}
\end{subfigure}
\begin{subfigure}{\figwidth}\centering
\includegraphics[height=\figheight]{sources/applications/house.png}
\caption{House}\label{applications_house}
\end{subfigure}
\begin{subfigure}{\figwidth}\centering
\includegraphics[height=\figheight]{sources/applications/pocket_operator_case.png}
\caption{Case}\label{applications_case}
\end{subfigure}
\begin{subfigure}{\figwidth}\centering
\includegraphics[height=\figheight]{sources/applications/david.png}
\caption{Statue}\label{applications_statue}
\end{subfigure}
\begin{subfigure}{\figwidth}\centering
\includegraphics[height=\figheight]{sources/applications/tud_logo.png}
\caption{TUD}\label{applications_tud}
\end{subfigure}
\begin{subfigure}{\figwidth}\centering
\includegraphics[height=\figheight]{sources/applications/ultimaker_logo.png}
\caption{UM}\label{applications_um}
\end{subfigure}
\begin{subfigure}{\figwidth}\centering
\includegraphics[height=\figheight]{sources/applications/pinion_gear_motor.png}
\caption{Gear}\label{applications_gear}
\end{subfigure}
\begin{subfigure}{\figwidth}\centering
\includegraphics[height=\figheight]{sources/applications/topopt_bone.png}
\caption{Bone}\label{applications_bone}
\end{subfigure}
\begin{subfigure}{\figwidth}\centering
\includegraphics[height=\figheight]{sources/applications/gyroid.png}
\caption{Gyroid}\label{applications_gyroid}
\end{subfigure}
\begin{subfigure}{\figwidth}\centering
\includegraphics[height=\figheight]{sources/applications/hex_grid.png}
\caption{Hex}\label{applications_hex}
\end{subfigure}
\caption{
Visualization of the widths for the output toolpaths of the inward distributed beading scheme applied to various example application objects.
A legend for the colors can be found in \cref{visualized_accuracy}.
From left to right and top to bottom: a house, a case for electronics, a statue, two common logos, a gear, a topologically optimized bone structure, a homogeneous lateral thickness tilted gyroid structure and a heterogeneous thickness hexagonal grid.
}
\label{applications_overview}
\end{figure*}




