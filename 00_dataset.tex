\documentclass[5p,twocolumn,10pt,times]{elsarticle}
\usepackage{amsmath}
\usepackage{hyperref} % added [draft] to avoid compilation issues that happen if a link is split and appears in two pages
%\modulolinenumbers[5]
\addtolength{\textheight}{8mm}
\addtolength{\textwidth}{4mm}
\addtolength{\voffset}{-10mm}
\addtolength{\hoffset}{-3mm}

\bibliographystyle{elsarticle-num-names}


% ACM template
%
%\documentclass[acmtog,anonymous,timestamp,review]{acmart}
%
%\usepackage{booktabs} % For formal tables
%




% My TK added packages and commands

	% for for using hyperref and elsarticle-num-names together in order to get \citeauthor to work
	\makeatletter
	\providecommand{\doi}[1]{%
	  \begingroup
	    \let\bibinfo\@secondoftwo
	    \urlstyle{rm}%
	    \href{http://dx.doi.org/#1}{%
	      doi:\discretionary{}{}{}%
	      \nolinkurl{#1}%
	    }%
	  \endgroup
	}
	\makeatother

	% have multiline subfigure captions be centered
	\usepackage[labelformat=parens]{subcaption} % subfigures
	\captionsetup[subfigure]{justification=centering}
	\captionsetup{subrefformat=parens} % pure refernce subfigure with parentheses: fig.10a and (b)
	%\renewcommand\thesubfigure{(\alph{subfigure})} % refernce subfigure always with parentheses: fig.10(a) and (b)

	\captionsetup[figure]{labelfont={bf},name={Fig.},labelsep=period} % use `Fig.' for figure subscript instead of `Figure'
	
	\usepackage[export]{adjustbox} % [right] alignment for includegraphics
	
	\usepackage{rotating} % turn env for rotating text in figures

	\usepackage{wrapfig} % inline figures

	% tables
	\usepackage{multirow} % multicolumn, multirow
	\usepackage{colortbl} % \cellcolor{<color>}
	\newcolumntype{C}[1]{>{\centering\arraybackslash}m{#1}}   %% centered
	\newcolumntype{R}[1]{>{\raggedleft\arraybackslash}m{#1}}  %% right aligned

	\usepackage[capitalise]{cleveref} % automatically add `Fig.'  etc before a reference.

        \usepackage{ amssymb } % \therefore
	
	\newcommand{\degree}{^\circ}
	
	\usepackage[binary-units]{siunitx} % mm and stuff
	\sisetup{per-mode = symbol}
	\DeclareSIUnit\pixel{px}

	\usepackage{units} % \nicefrac{3}{8}
	
	
	
	\DeclareMathOperator*{\argmax}{arg\,max}
	\DeclareMathOperator*{\argmin}{arg\,min}
	
	\DeclareMathOperator{\abs}{abs} % absolute function

	\usepackage{amsthm} % \begin{proof}
	\newtheorem{lemma}{Lemma}[section]
	\theoremstyle{definition}
	\newtheorem{definition}{Definition}[section]

	\usepackage[inline]{enumitem} % inline enumerate*

	\usepackage[toc,page]{appendix} % appendicces
	
	\usepackage{pgfplots}
	\usepackage{pgfplotstable} % tikzpicture table plots
	\pgfplotsset{compat=1.15}
	\usetikzlibrary{backgrounds}

	\usepackage[noend]{algpseudocode} % algorithmic
	\usepackage{algorithm} % wrapper for pseudocode to give a caption and label

	\newcommand{\pluseq}{\mathrel{+}=} %pluseq symbol
	\usepackage{upgreek} % \uplambda

	\usepackage{listings} % for listing C++ code instead of pseudocode
	\lstset{ 
      breaklines=true,                 % sets automatic line breaking
      basicstyle=\ttfamily,
      mathescape
    }




    % \usepackage[disable]{todonotes} % notes not showed  
    % \usepackage[draft]{todonotes}   % notes showed
    \usepackage{color,soul} % caps, highlight (\hl)

	\newcommand{\comment}[1]{}
	
    \newcommand{\todo}[1]{\hl{#1}}
    
	\newcommand{\temp}[1]{\textcolor[rgb]{0, 0, 0.2}{#1}}
	\newcommand{\tim}[1]{\temp{\todo{[Tim: #1]}}}
	\newcommand{\jun}[1]{\temp{\todo{[Jun: #1]}}}
	
	\newcommand{\old}[1]{\textcolor{gray}{#1}}
	\usepackage[normalem]{ulem}
	\newcommand{\stkout}[1]{\ifmmode\text{\sout{\ensuremath{#1}}}\else\sout{#1}\fi}
	
	% Revise macro (usage: \revise{old}{new})
	% Version a) First arg red and striked out, second argument green
	%\newcommand{\revise}[2]{\textcolor{red}{\stkout{#1}}\textcolor{blue}{#2}}
	%\newcommand{\revise}[2]{{\color{red}{#1}\color{blue}{#2}}}
	% Version b) First arg ignored, second argument green
	\newcommand{\revise}[2]{{\color{blue}{#2}}}
	% Version c) First arg ignored, second argument unchanged (for final draft)
	%\newcommand{\revise}[2]{#2}
	%\newcommand{\revise}[2]{#1}


	\setulcolor{red}

	\usepackage[normalem]{ulem} % squigly underline

	\renewcommand\floatpagefraction{.8}


	\newlength{\figwidth}
	\newlength{\figwidthTwo}
	\newlength{\figwidthTree}
	\newlength{\figheight}
	\newlength{\figheightTwo}
	\newlength{\tempheight}
	\newlength{\tempheightTwo}

	% deal with missing images which are not directly included in the repo
	\iffalse
	\newcommand{\noimage}[1]{%
	  \setlength{\fboxsep}{-\fboxrule}%
	  \fbox{\phantom{\rule{10pt}{10pt}} Missing file: \path{#1} \phantom{\rule{10pt}{10pt}}}% Framed box
	}
	\let\includegraphicsoriginal\includegraphics
	\renewcommand{\includegraphics}[2][width=\textwidth]{\IfFileExists{#2}{\includegraphicsoriginal[#1]{#2}}{\noimage{#2}}}

	\fi
% ENd of TK's added packages and commands



\begin{document}
\baselineskip11pt 

\begin{frontmatter} 

\title{Data set}

%\author{Paper ID: xxx}

\author[um,tud]{Tim Kuipers}
\address[um]{Ultimaker, Utrecht, The Netherlands}
\address[tud]{Department of Design Engineering, Delft University of Technology, The Netherlands}


%
% The code below should be generated by the tool at
% http://dl.acm.org/ccs.cfm
% Please copy and paste the code instead of the example below.
%
%\begin{CCSXML}
%\end{CCSXML}

%\ccsdesc[500]{Computer systems organization~Embedded systems}
%\ccsdesc[300]{Computer systems organization~Redundancy}
%\ccsdesc{Computer systems organization~Robotics}
%\ccsdesc[100]{Networks~Network reliability}


\end{frontmatter}




\section{Dataset}\label{dataset}
The dataset we tested on was a custom selected set of open source 3D models found on the internet which was selected to cover a broad range of different types of application and geometry.
Applications range from prototypes, to fixtures and mechanical end-use parts.
The geometry covers a wide range including thin filaments, smooth surfaces, organic shapes, chamfered shapes, small shapes and large shapes.
The models are described in \cref{dataset_description}.

\begin{table}
\caption{3D models used for validation}\label{dataset_description}
\begin{tabular}{l l l}
Name & Author & License \\
\hline
\href{	https://www.thingiverse.com/thing:26555	}{	AirCasting Air Monitor Casing	} & 	HabitatMap	 & 	CC-A	\\
\href{	https://www.thingiverse.com/thing:3629434	}{	Air hose splitter	} & 	frizinko	 & 	CC-A	\\
\href{	https://www.thingiverse.com/thing:1155772	}{	Al Hamra Tower	} & 	TurnerConstructionCompany	 & 	CC-A	\\
\href{	https://www.thingiverse.com/thing:1498967	}{	canon NP-E3 battery cap	} & 	kosuyoung	 & 	CC-A	\\
\href{	https://www.thingiverse.com/thing:3567409	}{	David	} & 	Thunk3D	 & 	CC-A	\\
\href{	https://www.thingiverse.com/thing:3132621	}{	Deck Assembly Tool, Plank Screwing Tool	} & 	PSomeone	 & 	CC-A	\\
\href{	https://www.thingiverse.com/thing:2920060	}{	Ender 3 Cable Chain	} & 	johnniewhiskey	 & 	CC-A	\\
\href{	https://www.thingiverse.com/thing:2993875	}{	Ergonomic Hacksaw Handle	} & 	mmOne	 & 	CC-A	\\
\href{	https://www.thingiverse.com/thing:2513922	}{	Gap measurement tool	} & 	ravm84	 & 	CC-A	\\
\href{	https://www.thingiverse.com/thing:1673030	}{	G-Clamp fully printable	} & 	johann517	 & 	CC-A	\\
\href{	https://www.youmagine.com/designs/gyroid	}{	Gyroid	} & 	Tim Kuipers	 & 	PD	\\
\href{	https://www.thingiverse.com/thing:1327093	}{	3D Printable Jet Engine	} & 	CATIAV5FTW	 & 	CC-NC	\\
\href{	https://www.thingiverse.com/thing:2752165	}{	Lawn Mower Throttle Replacement	} & 	Spammington	 & 	CC-A	\\
\href{	https://www.thingiverse.com/thing:2854328	}{	OpenRC F1 Internal gear box mod	} & 	intoxikated	 & 	CC-A	\\
\href{	https://www.thingiverse.com/thing:3592328	}{	PCB Test Fixture	} & 	JMadison	 & 	CC-A	\\
\href{	https://www.thingiverse.com/thing:1078865	}{	Pioneer Radio Holder for Ford Focus	} & 	Perugino	 & 	CC-A	\\
\href{	https://www.thingiverse.com/thing:34596	}{	Replicator Dual Fan Mount	} & 	aubenc	 & 	PD	\\
\href{	https://www.thingiverse.com/thing:3754728	}{	Atuador vers\~ao 2 * actuator version 2	} & 	Caroline Holanda	 & 	CC-A	\\
\href{	https://www.thingiverse.com/thing:1595179	}{	TE Pocket Operator Hard case	} & 	Salvation76	 & 	CC-A	\\
\href{	https://www.thingiverse.com/thing:3682303	}{	Screw sizer	} & 	Pierrolalune63	 & 	CC-NC	\\
\href{	http://homepage.tudelft.nl/z0s1z/projects/2017-bone-infill.html	}{	Bone-like optimized infill	} & 	Jun Wu	 & 	CC-A	\\
\href{	https://www.thingiverse.com/thing:26244	}{	Two-Story Spec House	} & 	pwc-phil	 & 	CC-A	\\
\end{tabular}
\end{table}


\end{document}
