\section{Conclusion and future work}

\begin{itemize}
\item Accurate filling using few/small overfill and underfill areas
\item Which allows for accurate reproduction of thin parts and variable thickness microstructures.
\item Our framework is able to emulate several methods proposed in related literature.
\item Our framework allows for deposition width control which reduces the amount of extrusion paths with wildly different width compared to previous methods
\item Our framework allows for limiting the stretch on the outer toolpath in order to reduce any adverse affect of variable extrusion width on dimensional accuracy.
\item Our framework is stable against slight perturbations of the outline shape or large perturbations of the outline shape in far away regions.
\item We can define a strategy which distributes the bead width deviation over a small region so as to retain smooth toolpaths of nominal width throughout the larger part of big regions.
\item Our strategy keep mechanical properties more constant.
\item The computation times are approximately 3(?) times as fast as the state of the art method for naive toolpath generation employing constant width.
\end{itemize}



\subsection{Future work}
Define the optimal strategy for a given piece of hardware by finding an accurate objective function which takes into account the manufacturability or dimensional accuracy as a function of bead width and its effect on qualities of top layers, stiffness or the dimensional accuracy of the outline shape.

Use Skeletal Trapezoidation to generate quad dominant meshings for FEM.

Generate thicker bead widths near sharp corners in order to mimic limitations in the extrusion process.
Allow for adjusting the toolpath widths to local part curvature.

Allow for information sharing between the layers.