\section{Introduction}
3D printing enables the fabrication of complex geometry under few design constraints compared to conventional fabrication techniques.
Recent developments have seen a rapid growth in both the use and capabilities of desktop 3D printing systems.
%The rapid spread of 3D printing through different industries and types of application calls for the possibility to manufacture a wide range of geometries while guaranteeing mechanical properties of the resulting parts.
Fused Deposition Modeling (FDM) is one of the most common 3D printing techniques.
It is widely used because of the versatility in the types of plastic which can be used and the relatively low running costs.
FDM printers are used, for example, in showcasing scale models of buildings, casings for electronics, prototypes for blow molded parts, jigs and fixtures.
Recent research developments have investigated manufacturing complex volumetric structures such as microstructures~\cite{bates2018compressive,Al-Ketan2018,Maskery2018} and topology optimized structures~\cite{Zegard2016SMO,Wu2019a,Cheng2019}.
Many of these applications involve 3D models with detailed features within the order of magnitude of the printing resolution, which restrains the field of the process planning algorithms.

FDM printers extrude semi-continuous beads of molten plastic through a nozzle, which moves along a planned toolpath within each layer of a 3D object.
A common strategy to accurately manufacture a given 3D model is to extrude along a contour-following path,
because the position and shape of the toolpath can be controlled relatively accurately.
Filling up a shape using parallel straight lines would expose defects of the size of the hole in the nozzle, which is generally an order of magnitude larger than the resolution of the positioning system.
Contour-parallel extrusion therefore leads to a less bumpy outline shape than direction-parallel extrusion does.


The simple technique for generating the dense contour-parallel toolpaths of a layer consists in performing uniform inward offsets with the size of the nozzle from the outline shape.
However, for geometrical features which are not an exact multiple of the nozzle size this method produces areas where an extrusion bead is placed twice: \emph{overfill} areas; and areas which are not filled at all: \emph{underfill} areas.
See \cref{intro_wedge_uniform}.
Overfills cause a buildup of pressure in the mechanical extrusion system, which can result in bulges or even a full print failure.
Underfills on the other hand, can cause a drastic decrease in the part stiffness or even for small features not to be printed at all.
These problems are exacerbated for models with layer outlines with small features, because the over- and underfill areas are relatively large compared to the whole part.

One promising direction to avoid over- and underfills is to employ toolpath with adaptive width.
Wire and arc additive manufacturing supports a wide range of widths~\cite{Ding2014,Xiong2019},
which led \citeauthor{Ding2016a} to develop a toolpath strategy which produces a width variation typically lower than a factor of $3$, but is sometimes far greater~\cite{Ding2016a}.
However, the limitations on width imposed by the hardware system are far greater for FDM.
A nozzle of \SI{0.4}{\milli\meter} will typically start to cause fluttered extrusion around lines narrower than \SI{0.3}{\milli\meter} and lines will start to bulge upward if they are wider than the flat part of the nozzle, which is typically \SI{1.0}{\milli\meter}.
%Therefore, a limited range of widths is required by the hardware system.

The current state-of-the-art for FDM printing developed by \citeauthor{Jin2017JMS} is employing a strategy which alters the widths of the centermost beads at most by a factor of $2$~\cite{Jin2017JMS}.
See \cref{intro_wedge_centered}.
Still, controlling the extrusion width through movement speed changes or through volumetric flow control (e.g. linear advance) yields diminishing accuracy for deposition widths farther away from the nozzle size,
since process parameters such as nozzle temperature are optimized for beads with the nozzle size.
Moreover, reducing the variation in width is beneficial for limiting the variation in mechanical properties of the resulting product, meaning it conforms better to a simulation which employs a heterogeneity assumption.
Therefore, a narrower range of widths is desirable.

In this paper we propose a framework for planning toolpaths with control over the adaptive width for minimizing over- and underfill. 
We show that this framework supports various control schemes for determining the bead spacing and extrusion widths. 
For FDM printing in particular we propose a novel scheme which reduces the amount of over- and underfill while ensuring the extrusion beads deviate little from the nozzle size.
See \cref{intro_wedge_distributed}. 
% In order to avoid over- and underfills in regions which are not as wide as an exact multiple of the nozzle size we need algorithms which produce toolpaths which employ adaptive bead width.
% Some such toolpath generation techniques have already been developed.


\begin{figure}\centering
\setlength{\figwidth}{.9\columnwidth}
\setlength{\figwidthTwo}{.05\columnwidth}
\begin{subfigure}{\figwidth}\centering
\parbox[b]{\figwidthTwo}{\subcaption{}\label{intro_wedge_uniform}}\includegraphics[width=\figwidth]{sources-intro-TEST-naive-accuracy.png}
\end{subfigure}
\begin{subfigure}{\figwidth}\centering
\parbox[b]{\figwidthTwo}{\subcaption{}\label{intro_wedge_centered}}\includegraphics[width=\figwidth]{sources-intro-TEST-Center-widths.png}
\end{subfigure}
\begin{subfigure}{\figwidth}\centering
\parbox[b]{\figwidthTwo}{\subcaption{}\label{intro_wedge_distributed}}\includegraphics[width=\figwidth]{sources-intro-TEST-InwardDistributed-widths.png}
\end{subfigure}
\caption{
Illustration of different toolpath for a wedge shape.
\subref{intro_wedge_uniform} Toolpath using uniform offsetting results in large overfill (orange) and underfill (azure).
\subref{intro_wedge_centered} Toolpath with adaptive width~\cite{Jin2017JMS} where beads that are wider or narrower than the nozzle size are indicatd in red and blue, respectively.
\subref{intro_wedge_distributed} Our approach minimizes over- and underfill with beads close to the nozzle size.
}
\label{intro_wedge}
\end{figure}


Our contributions are as follows:
\begin{itemize}
\item A geometric framework for generating densely filling contour-parallel toolpaths employing adaptive width, according to any beading scheme which decides on the bead spacing and widths.
\item A specific beading scheme for FDM printing which reduces the amount of deviation in the extrusion widths, and which promotes smooth toolpaths that are equal to the preferred width toward the outline of the shape.
\end{itemize}



%This work is patent pending, but the source code is available open source.

